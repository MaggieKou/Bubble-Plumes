\documentclass{article}
\RequirePackage{graphicx} % Required for inserting images
\RequirePackage{natbib}
\RequirePackage{amsmath,amsfonts,amssymb}
\RequirePackage{graphicx,xcolor}
\RequirePackage{gensymb}
\usepackage{todonotes}
\usepackage{hyperref}
\title{Subglacial Plume vs Bubbles - Tabletop Lab Experiment Proposal V1}
\author{Maggie Kou}
\date{October 2025}

\begin{document}

\maketitle

\listoftodos
\section{Introduction}
Subglacial plumes occur at the front of ice shelves. The vertical length scales for these ice shelves can range from 100 m to 2 km \citep{jenkins_observation_2010}.
The turbulent plume forms from two buoyant sources: 1) submarine ice melt and 2) subglacial discharge. In both cases, the rejected liquid water is near fresh with a density deficit $\approx 0.028$ \citep{hewitt_subglacial_2020}. For melting, energy required also cools the surrounding waters with density deficit $ \approx 0.024$. These are represented in \ref{fig:gen-diagram-hewitt} and labelled $\dot m $ and $q_{sg}$ respectively.

\begin{figure} [h]
    \centering
    \includegraphics[width=\linewidth]{Hewitt-fig3.png}
    \caption{A figure of the presumed salinity and temperature profiles next to the ice shelf and plume. b) shows typical oceanic profiles; c) shows typical plume profiles overlayed in lighter colours}
    \label{fig:gen-diagram-hewitt}
\end{figure}

High molecular Pr (13.8) and Sc (2432) for ocean water means a thin viscous boundary layer within the thermal and saline boundaries where temperature and salinity experiences the largest variation \citep{steele_role_1989}, consistent with lab experiments, and general models for buoyant plumes rising next to vertical plates \cite{wells_geophysical-scale_2008}. 

Observations by \citep{shepherd_warm_2004} indicate that ablation rate may be linear in $\triangle T$ (the temperature difference between the ambient  temperature and the liquidus temperature.  
\cite{mcphee_revisiting_2008} present measurements of turbulent heat and salinity at Svalbard fjord under growing ice and compare to a model of double diffusion with $R$, a ratio between the interfacial heat exchange coefficient and the interfacial salt coefficient. 

For homogeneous ambient salinity and temperature, both the theory and experiments of \cite{kerr_dissolution_2015, mcconnochie_effect_2016, mcconnochie_turbulent_2016, mcconnochie_enhanced_2017} showed that the ablation velocity and interface temperature are uniform with height once the wall plume is turbulent (heights $\geq$ 10 cm) while plume velocity increases with height to the 1/3 power. Equivalent results were obtained by \cite{gayen_simulation_2016}. 

\subsection{Key Parameters and Models}

\subsubsection{Constitutive Relations}
Equation of state for the density of the ocean water is usually given by a linear approx:
\[\frac{\rho_a - \rho}{\rho_o} = \beta_S (S_a - S) - \beta_T (T_a - T)\]
Where the subscript a denotes a quantity of the ambient conditions. $\beta_S$ and $\beta_T$ are expansion coefficients. Salinity exerts primary control on buoyancy since $T_a - T$ is very low \citep{hewitt_subglacial_2020}.

The freezing temperature is given by a linear liquidus function:
\[T_L(S,z) = T_o + \lambda z - \Gamma S\]
Where $T_o$ is a reference temperature, $\lambda$ is the slope of the freezing point depth, and $\Gamma$ is the saline dependence \citep{hewitt_subglacial_2020}.

\subsubsection{Key Nondimensional Numbers}
For laminar boundary layers next to a flat plate, the thickness of the velocity, thermal, and saline boundary layer thicknesses are given \citep{schlichting_boundary-layer_2017} in terms of the Prandtl ($Pr = \frac{\nu}{\kappa_T}$), Schmidt ($Sc = \frac{\nu}{\kappa_S}$, and Lewis ($Le = Sc/Pr$) numbers respectively: 
\[\frac{\delta_T}{\delta_u} \simeq Pr^{-1/2} \approx 0.3\]
\[\frac{\delta_S}{\delta_u} \simeq Sc^{-1/2} \approx 0.02\]
\[\frac{\delta_S}{\delta_T} \simeq Le^{-1/2} \approx 0.07\]
Where estimates are given by \cite{josberger_laboratory_1981}. 

Grashof Number number for when the buoyant plume becomes turbulent:
\[Gr = \frac{g (\rho_w - \rho_{\infty})l^3}{\rho_{\infty} \nu^2} \geq 10^9\]
This happens at length scales greater than 0.5 m for oceanic salinity and temperatures \citep{josberger_laboratory_1981}.

The Nusselt number is: \[Nu = \frac{qH}{k(T_w - T_f)}\] Where $T_w$ is the temperature of the wall, $T_f$ the far-field temperature of the fluid, $q$ the heat flux from the boundary with height $H$, and $k$ is the thermal conductivity of the fluid. 

The Rayleigh number can also be written as:
\[Ra = \frac{g \alpha H^3 (T_w - T_f)}{\kappa \nu}\]
Where $g$ is the gravitational constant, $\alpha$ us the thermal expansion coefficient, $\kappa$ is the thermal diffusivity, and $\nu$ is the kinematic viscosity. 

Empirically, \cite{warner_experimental_1968} find that:
\[Nu = 0.10 Ra^{1/3}\]

The Stefan number provides a ratio of the sensible heat to the latent heating:
\[Sf = \frac{\rho_s L_s}{\rho_w c_w(T_w - T_i)}\]

The Stanton numbers correspond to the transfer coefficients, given by \citep{jenkins_observation_2010, hewitt_subglacial_2020} as:
\[St_T = C_d^{1/2} \Gamma_T\]
\[St_S = C_d^{1/2} \Gamma_S\]

\cite{mcphee_revisiting_2008} define a lumped Stanton number to approximate the melting equations with limited constraints:
\[St = C_d^{1/2} \Gamma_{TS}\]

Some literature \citep{gayen_simulation_2016} use the buoyancy ratio to denote the contributions between salinity vs temperature: 
\[R_{\rho} = \frac{\beta_S (S_w - S_i)}{\beta_T (T_w - T_i)}\]
\subsubsection{Melt/Dissolution Models}

\cite{hewitt_subglacial_2020} define a modified latent heat: 
\[\tilde{L} = L + c_i (T_L(S_i)-T_i)\]
Which accounts for both release of latent heat from phase transition, and all the energy involved in warming the ice to melting temperature. 
Since after melting, the fresh water will have some buoyancy relative to its surroundings, this can be written as an 'effective' temperature excess:
\[\triangle T_i^{ef} = -\frac{\tilde L}{c}\]

Finally, we may note that the boundary is in local equilibrium: 
\[T_b = T_L(S_b)\]

Starting with an energy balance at the phase change interface: 
\[\dot m L  \; + \; \dot mc_i(T_b - T_i) =  St_T U c(T-T_b) \]
Where the second term is given by the convective mixing by the turbulent plume the transfer coefficient is given by $St_T U$ where the thermal Stanton number is constant, $c$ is the heat capacity of the ocean water, and $U$ is the velocity of the plume (although sometimes includes tidal forcing) \cite{jenkins_observation_2010}. 

Similarly, for salinity we can write: 
\[\dot m (S_b - S_i) = St_SU(S-S_b)\]

Alternatively, \cite{mcphee_revisiting_2008} propose a reduced model by take the temperature at the boundary to be the freezing point slightly outside the boundary (and with the lumped Stanton number). If one assumes that $T_L(S) - T_L (S_i) << \frac{\hat L}{c}$ i.e. that the difference in freezing temperatures of the ice and liquid is much less than the excess temperature, then we can write: 
\begin{align}
    \boxed{\dot m = St U c\frac{T - T_L(S)}{\tilde L}} \label{eq:melt-rate-simple}
\end{align}
or, a direct ratio of the thermal driving with the temperature excess. 

\subsubsection{Plume Models}
\cite{wells_geophysical-scale_2008} consider the formation of a turbulent plume off a vertical plate and describe a laminar region, a turbulent region with a laminar sublayer set by a balance between the viscosity and the buoyancy, and the outer plume by inertia and buoyancy, and finally, the top-most region where the sublayer width is controlled by the shear imposed by the turbulent plume. They give lengthscales and corresponding $Re_z$ and $Nu$ thresholds.

Generally, once a developed, turbulent plume is reached, \cite{hewitt_subglacial_2020} write the mass, momentum, salt, and heat balances as: 
\begin{align}
    \frac{\partial}{\partial X}(DU) &= \dot e + \dot m \label{eq:plume_massbal}\\
    \frac{\partial}{\partial X}(DU^2) &= D  \frac{\triangle \rho}{\rho_o} g \sin \phi - C_d U^2 \label{eq:plume_momentumbal}\\
    \frac{\partial}{\partial X}(DUS) &= \dot e S_a+ \dot m S_i \label{eq:plume_salbal}\\
    \frac{\partial}{\partial X}(DUT) &= \dot e T_a + \dot m T_i^{ef} \label{eq:plume_heatbal}
\end{align}
Where $D$ is the thickness of the plume, $\triangle \rho = \rho_a - \rho$ is the buoyancy or density deficit relative to ambient, $\dot e = E_o \sin \phi \; U$ us the assumed entrainment rate. $\dot m$ is given by Equation \ref{eq:melt-rate-simple}. Usually the melt rate has little mass contribution when compared to the entrainment rate but is significant in delivering cold freshwater for heat and salt balances and conversely the entraiment rate has little contribution to density. We can re-write everything in terms of density and temperature excess where we have substituted in Equation \ref{eq:melt-rate-simple}:

\begin{align}
    \frac{\partial}{\partial X}(DU) &=  E_o \sin \phi \; U\label{eq:simplified_massbal}\\
    \frac{\partial}{\partial X}(DU^2) &= D  \frac{\triangle \rho}{\rho_o} g \sin \phi - C_d U^2 \\
    \frac{\partial}{\partial X}(DU \triangle \rho) &= St U \frac{c \triangle T}{\tilde L} \triangle \rho_i^{ef}+ \sin \phi \frac{\mathrm{d} \rho_a}{\mathrm{d} z} D U \label{eq:buoyancy}\\
    \frac{\partial}{\partial X}(DU \triangle T) &= E_o \sin \phi \; U \triangle T_a + St U \frac{c \triangle T}{\tilde L} \triangle T_i^{ef} - \lambda \sin \phi DU\label{eq:temp-deficit}
\end{align}

\subsection{Scaling and Limits}

From Equation \ref{eq:simplified_massbal}, if we take the ambient conditions to be more or less uniform, we can assume a length scale for the flows: $D \approx E_o \sin \phi \; x$. If $\partial_X (DU\triangle \rho) \approx 0 $ or there is a negligible effect of adding meltwater as done by \cite{jenkins_convection-driven_2011} Equation \ref{eq:buoyancy} gives:
\[ \triangle \rho' = \triangle \rho_{sg} \frac{q_{sg}}{D U}\]
and similarly from \ref{eq:temp-deficit} we have:
\[\triangle T' = \frac{E}{E + St} \triangle T_{a0}\]
Where $\triangle T_{a0}$ is the initial ambient driving temperature.

This gives the velocity constant:
\[U' = (\frac{q_{sg}\triangle \rho_{sg} g \sin \phi }{\rho_o (E+C_d)} )^{1/3}\]
Where $q_{sg}\triangle \rho_{sg}$ is given as the buoyancy flux from the subglacial discharge. As discussed by \cite{hewitt_subglacial_2020}, the subglacial flux has little contribution to the mass balance when compared to the entrainment so the flow behaves similar to an ideal plume. 

In the limit of uniform ambient quantities, these exact solutions provide the following melt rate:
\begin{align}
    \boxed{\dot m = \frac{ESt}{E+St}\frac{c \triangle T_{a0}}{\tilde L }(\frac{q_{sg}\triangle \rho_{sg} g \sin \phi }{\rho_o (E+C_d)} )^{1/3}}
\end{align}

A further length scale can be found by including the buoyancy term from Equation \ref{eq:buoyancy}:
\[l_{\rho} = \frac{1}{\sin \phi}\left(\frac{E+C_d}{E}\right)^{1/2}\left|\frac{g}{\rho_o}\frac{\mathrm{d}\rho_a}{\mathrm{d}z}\right|^{-1/2}(\frac{q_{sg}\triangle \rho_{sg} g \sin \phi }{\rho_o (E+C_d)} )^{1/3}\]
This is an important length scale where the buoyancy flux is lost to ambient stratification.

Which then with the transformations $X = l_{\rho} \hat X$, $D = E l_{\rho} \hat D$, $U = U' \hat U$, and $\triangle \rho = \triangle \rho' \triangle \hat\rho$, $\triangle T = \triangle T' \triangle \hat T$ yield:

\begin{align}
    \partial_{\hat X}(\hat D \hat U) &= \hat U \\
    \partial_{\hat X}(\hat D \hat U^2) &= (1 + \hat{C_d})\hat D \triangle \hat \rho -  \hat{C_d} \hat U^2\\
    \partial_{\hat X}(\hat D \hat U \triangle \hat \rho) &= - \hat D \hat U \\
    \partial_{\hat X}(\hat D \hat U \triangle \hat T) &= (1+\hat St)\hat U \triangle \hat T_a - \hat St \hat{\dot m}
\end{align}
Where $\hat C_d = C_d/E$, $\hat St = St/E$, $\triangle \hat T_a = \triangle T_a/T_{a0}$, and $\hat{\dot m} = \hat U \triangle \hat T$. 

Simulations from \cite{hewitt_subglacial_2020} show that $X_{neg} \approx 1.44 l_{\rho}$ is when the plume becomes negatively buoyant and $X_{stop} \approx 2.06 l_{\rho}$ is when it comes to rest. 

Another lengthscale $l_{sg}$ characterises when the buoyancy flux of submarine melting becomes more important than the subglacial discharge: 
\[\boxed{l_{sg} = \frac{q_{sg} \triangle q_{sg}}{\dot m \triangle \rho_i^{ef}} = \left( \frac{\rho_o (E+C_d)}{g \sin \phi}\right)^{1/3}\left( \frac{E + St}{ESt}\right)\frac{\tilde L \triangle \rho_{sg}^{2/3} q_{sg}^{2/3}}{c \triangle T_{a0} \triangle \rho_i^{ef}}}\]

Finally, the depth dependence of the freezing point is significant on the timescale: 
\[\boxed{l_T = \frac{\triangle T_{a0}}{\lambda \sin \phi}}\]
\subsection{Previous Experiments}
\cite{josberger_laboratory_1981} ran a lab experiment of melting ice walls in uniform far field salinity ($S_{\infty}$) and temperature ($T_{\infty}$), for $30 \% \leq S_{\infty} \leq 35 \%$ and $T_{\infty} \leq 20 \degree C$ (akin to bulk Oceanic conditions), they find that the boundary flow develops a bidirectional laminar flow from varying salinity and temperature boundary layer sizes which allows the inner to transition into a turbulent plume and the outer to sink to the bottom. 
\cite{huppert_ice_1980} conduct a similar experiment, at a shorter scale, to investigate an ice block melting in a stably stratified ambient fluid, finding that the melt water spreads in nearly horizontally stratified layers. 

\cite{cenedese_impact_2016} conduct an experiment to study the heat flux and melting of an ice block in a two-layer stratified ambient fluid with a point source plume near the bottom. They found that melting increased with the strength of subglacial discharge. \cite{kerr_dissolution_2015} test the dissolution of a vertical ice wall by the turbulent plume generated from melting/dissolution only with a homogenous background fluid. In \cite{mcconnochie_effect_2016} the same experiments were repeated but for a stable salinity gradient in the far field, finding double diffusive patterns akin to \cite{huppert_ice_1980}. \cite{mcconnochie_enhanced_2017} study the effect of a freshwater plume near the ice face and find that the ablating velocity increases with the buoyancy flux of the source. 
\section{Experimental Designs}
\subsection{Control Tests}
These series of controlled tests are designed to illustrate the melting and heat/mass transport by the turbulent plume only (in the natural ocean context). In general, I am interested in measuring: 
\begin{enumerate}
    \item temperature at various heights at the interface.
    \item the temperature of the bulk ambient fluid.
    \item the salinity in the bulk ambient fluid. 
    \item the salinity near the mixing region.
    \item the shape and size of the turbulent plume, perhaps a visualization of the surrounding entrained fluid.  
    \item the average velocity of the plume. 
\end{enumerate}
\subsubsection{Uniform $T_{\infty}$, $S_{\infty}$, No Subglacial Discharge}
This test will mimic \cite{kerr_dissolution_2015} with a uniform ambient salinity and temperature. Key measurements are 1) the rate of ice dissolution/melting, the temperature at the boundary as this happens, and the turbulent plume's corresponding velocity and shape. Notably, none of the past experiments seem to measure the continuous density distribution. It would be nice to estimate a length scale for entrainment as in \cite{bacot_bubble_2022}.

\subsubsection{Two-tier stratified $T_{\infty}$, $S_{\infty}$, No Subglacial Discharge}
Potentially skip but just as an analog to the Greenland fjords and to reference with \cite{cenedese_impact_2016}.
\subsubsection{Stratified Ambient Dissolution and Melting only}
Now no subglacial plume source, to simulate something along the lines of \cite{mcconnochie_effect_2016}.
\subsubsection{Dissolution and Melting with Subglacial Plume}
Now with apparatus like \cite{cenedese_impact_2016}. \\
Vary:
\begin{enumerate}
    \item subglacial discharge strength
    \item stratification strength (density) - likely temperature first
    \item ice topology (perhaps shallower shelf extents which has not been done by the experiments of \cite{mcconnochie_enhanced_2017} etc... 
\end{enumerate}

\subsection{Full Experiments}
Finally, it's time to add a secondary barrier plume! The main parameters to adjust are: 
\begin{enumerate}
    \item strength of plume
    \item normal distance from the ice wall 
    \item size of bubbles
\end{enumerate}
Further, I would like to image the extent of this plume/mixing with the near-shelf water as well. 

\subsection{Future experiments}
The lengthscale of these flows are often large enough that they experience Coriolis effects. 

\section{Experimental Apparatus}
\subsection{General Apparatus}
\cite{kerr_dissolution_2015} use a  JulaboFP50 Refrigerated–Heating Circulator to rapidly cool the water and freeze into ice within the tank (then pumping excess liquid out/new liquid in). 

\cite{kerr_dissolution_2015} also use a 20 mm thick layer of acrylic with a 2 mm thick outer layer, separated by a 18 mm air gap to insulate the ice. 
\subsection{Oceanic Conditions}
For a set of experiments, a stably stratified salinity distribution and varying temperature distribution is required for the ambient surrounding waters. This can be accomplished with a two-bucket drip method as outlined by \cite{oster_density_1963}, or as a series of horizontally stratified layers with the spinning plate apparatus in the lab at DAMTP.
\subsection{Ice}
The ice will need to be produced and frozen without bubbles and in a large block (>1 m tall). This should be sorted out with the lab people. 
\todo{Speak with }
\subsection{Bubble Curtain}
Nozzle, adjustability for distance away from the ice shelf. 
\subsection{Measurement}
\begin{enumerate}
    \item Temperature measurement - in the ice, at the surface, and some distance from the plume; various heights. \cite{cenedese_impact_2016} use TenHOBO Pro v2 temperature dataloggers.
    
    \item Salinity - sampled at various heights outside of the plume with a traversing conductivity-temperature probe

    \item Ablation rate - usually done by taking two adjacent pictures, \cite{mcconnochie_effect_2016} take them 25 - 50 mins apart with repetition for the variance

    \item Density - \cite{bacot_bubble_2022} use a Anton PAAR DMA5000 density meter and dye attenuation with 
    \href{https://www.dalzielresearch.com/}{DigiFlow}.
\end{enumerate}
\subsection{Visualization}
\begin{enumerate}
    \item camera
    \item dye, potassium permanganate crystals and streaking
    \item \cite{mcconnochie_enhanced_2017} use a PIV system for visualizing the plume velocity. 
\end{enumerate}

\bibliographystyle{apalike}
\bibliography{references}
\end{document}
